%\documentclass[a4paper,11pt]{memoir} 
\documentclass[oneside, a4paper,11pt]{memoir} 

\usepackage[utf8]{inputenc}
\usepackage[danish]{babel}
\usepackage[T1]{fontenc}
\usepackage[left=4.0cm, right=2.0cm, top=3.0cm, bottom=3.0cm]{geometry} 
\usepackage{amsmath,amssymb}																						

% SIUnitX  http://ctan.org/pkg/siunitx
\usepackage{siunitx,booktabs}		
\sisetup{per=slash}																						
\sisetup{per-mode = reciprocal}
\sisetup{inter-unit-product = \ensuremath{{}\cdot{}}}
\sisetup{output-decimal-marker = {,} }
\DeclareSIUnit{\kroner}{kr.}
\DeclareSIUnit{\LSb}{LSb}
\DeclareSIUnit{\cycles}{cycles}
\DeclareSIUnit{\div}{Div}

\usepackage{graphicx}
\newsubfloat{figure}% Allow subfloats in figure environment
\usepackage{dcolumn,booktabs}
\usepackage{url}
\usepackage{wrapfig}

% Redigere billed-/tabelteksterne.
\usepackage{caption}
\usepackage{subcaption}
\captionsetup{font=small,
labelfont={it,bf},textfont=sf,
format=hang}

% Packages for color handling
\usepackage[usenames,dvipsnames,svgnames,table]{xcolor}

\usepackage{threeparttable}

\usepackage{lscape}

\usepackage{enumitem}

% TODONotes http://ctan.org/pkg/todonotes
\usepackage{todonotes}
\usepackage{placeins}
\usepackage{lastpage}

\usepackage[hidelinks]{hyperref}  
\hypersetup{bookmarks=false}
\hypersetup{pdftitle={Embedded Modul\ae rt Audio Effekt System}} 
\hypersetup{pdfsubject={4. Semester Projekt, F18, Grp. [3]}}
\hypersetup{pdfauthor={J\"{o}rn Jacobi, Emil Schier Christiansen, Henrik Lund Hansen, Jes Rydall Larsen, Sonny Fink, Frederik Halling}}


% For includering af .pdf
\usepackage{pdfpages}

% Bibliografi
\usepackage{babelbib}
\bibliographystyle{abbrv}

% Noget forsideopsætning
\usepackage{soul} % lege lege
\sodef\an{}{0.2em}{.9em plus.6em}{1em plus.1em minus.1em}
\newcommand\stext[1]{\an{\scshape#1}}

% New commands 
\newcommand{\g}{9,82 \si{\meter\per\second\squared}}
\newcommand{\dcite}[1]{\quotedblbase{#1}\textquotedblright}
\newcommand{\husk}[2]{\todo[inline,color=green!40]{#1: #2}}
\newcommand{\jj}[1]{\todo[inline,color=green!40]{JJ: #1}}
\newcommand{\note}[1]{\todo[inline]{#1}}
\DeclareMathOperator{\lapl}{\mathcal{L}}

% Remove paragraph indentation for document
\setlength{\parindent}{0pt}
\newcommand\hcancel[2][black]{\setbox0=\hbox{$#2$}%
	\rlap{\raisebox{.45\ht0}{\textcolor{#1}{\rule{\wd0}{1pt}}}}#2} 

% Listings package
\usepackage{listings}

%Fede overskrifter
\usepackage{kpfonts}
\usepackage{calc}
\setSingleSpace{1.0}
\SingleSpacing
\definecolor{chaptercolor}{gray}{0.8}
% helper macros
%\newcommand\numlifter[1]{\raisebox{-2cm}[0pt][0pt]{\smash{#1}}}
\newcommand\numlifter[1]{\raisebox{-.8cm}[0pt][0pt]{\smash{#1}}}
\newcommand\numindent{\kern37pt}
\newlength\chaptertitleboxheight
\makechapterstyle{hansen}{
  \renewcommand\printchaptername{\raggedleft}
  \renewcommand\printchapternum{%
    \begingroup%
    \leavevmode%
    \chapnumfont%
    \strut%
    \numlifter{\thechapter}%
    \numindent%
\endgroup%
}
  \renewcommand*{\printchapternonum}{%
    \vphantom{\begingroup%
      \leavevmode%
      \chapnumfont%
      \numlifter{\vphantom{9}}%
      \numindent%
      \endgroup}
    \afterchapternum}
  \setlength\midchapskip{0pt}
  \setlength\beforechapskip{0.5\baselineskip}
  \setlength{\afterchapskip}{1\baselineskip}
  \renewcommand\chapnumfont{%
    %\fontsize{3cm}{0cm}%
    \fontsize{2cm}{0cm}
    \bfseries%
    \sffamily%
    \color{chaptercolor}%
  }
  \renewcommand\chaptitlefont{%
    \normalfont%
    %\huge%
    \LARGE%
    \bfseries%
    \raggedleft%
  }%
  \settototalheight\chaptertitleboxheight{%
    \parbox{\textwidth}{\chaptitlefont \strut bg\\bg\strut}}
  \renewcommand\printchaptertitle[1]{%
    \parbox[t][\chaptertitleboxheight][t]{\textwidth}{%
      %\microtypesetup{protrusion=false}% add this if you use microtype
      \chaptitlefont\strut ##1\strut}%
}}
\chapterstyle{hansen}
\aliaspagestyle{chapter}{empty} % just to save some space

%linje afstand
%\DisemulatePackage{setspace}
%\usepackage[nodisplayskipstretch]{setspace}
%\setstretch{0.5}
%\siglespacing
%\onehalfspacing                                       
%\doublespacing

%compile debug, check pdflatex time
\newcommand\showtimer{Timer: \the\numexpr\the\pdfelapsedtime*1000/65536 \relax}
%\pdfresettimer}
%\usepackage{fancyhdr}
%\pagestyle{fancy}
%\fancyfoot[CE,CO]{\showtimer}


