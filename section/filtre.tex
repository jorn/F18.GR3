%	\item Begrundelse baseret på antal op-amp / orden (6/8)
%	\item Kort teoretisk gennemgang af den valgte topologi
%	\item Beregninger og dimensionering af 6. ordens chebycheb filtre
%	\item Implementeringsvalg og design af hardware/schematics
%\end{itemize}


\section{Analoge filtres rolle ved digital signalbehandling}\label{sec:filter_intro}
\note{Kort teoretisk intro til hvorfor filtre skal anvendes til sampling}
\note{Båndbrede begrænsning}
\note{Shannons sampling theorem}
\note{Generel intro til audio signaler}

\section{Analyse af filter typer}\label{sec:filter_topologier}
\note{Gennem gang af de filter topologier der har været i betragtning som en mulig løsning.}
For at kunne finde frem til et passende lavpasfilter som antialiasing filter, bliver 4 tilter typer sammenlignet - Butterworth, Chebyshev type I og II samt Bessel (Thomson).
Hver af disse filtre har en tilhørende amplitudekarakteristik  der beskrives med følgende overføringsfunktion.
\jj{kilde henvisning til afs note}

\begin{align} 
H_{butterworth}(j\omega_n) &= \frac{1}{\sqrt{1 + \omega_n^{2n}}}  \\
H_{chebychevI}(j\omega_n) &= \frac{1}{\sqrt{1 + \epsilon^2 C_n^2(\omega_n)}} \quad, C_n(\omega_n)\quad = \quad 
\begin{matrix}
\cos(n\arccos(\omega_n)) & 0 \le \omega_n \le 1 \\  \cosh(n \arccosh(\omega_n)) & 1 \le \omega_n 
\end{matrix}\\
H_{chebychevII}(j\omega_n) &= \frac{1}{\sqrt{1 + \frac{1}{\epsilon^2 C_n^2(\omega_n)}}} \quad\\
H_{bessel} (j\omega_n) &= \frac{H}{a_0 + \omega^2 + ja_1\omega_n}
\end{align}

\jj{renskriv Cn param til type I og II så det ser lidt pænere ud}

I fig. xx ses en samlet plot af $H(j\omega)$ 
\jj{indsæt graf af normeret sammenligning}

Det ønskede filter skal have en rimelig flad karakteristisk i pasbåndet og en stejl overgang til stopbåndet.
Således kan lydsignalet holdes konstant i hele det ønskede frekvensområde da uønsket dæmpning/forstærkning af lydsignalet kan medfører hørbare ændringer af lydbilledet. 
Ligeledes ønskes en rimelig konstant gruppeløbetid der er defineret som \cite{anfilter} 
\begin{align}
D(\omega) \stackrel{def}{=} - \dfrac{d(arg(N(\omega)))}{d\omega}\label{eq:groupdelay_def}
\end{align}
Gruppeløbetiden er et udtryk for hvor stor tidsforsinkelsen er på lydsignalet ved en given frekvens.

\note{Gennemgang af fordele/ulæmper ved filter typerne}
\note{endelig valg og begrundelse}
\note{Beregninger og argumentation for SNR}

\section{Specifikation og dimensionering}\label{key}
\note{sprecifikation som bi-quad systemer}
\note{hvor tæt ligger de beregnede komponent værdier i hold til de anvendte tabelværdier ?}
\note{Sallen Kay - metode 4 i afs matr.}
\note{Hvordan kommer aliasing til at have indflydelse på det endelige signal ved valg af 6. ordens filter} 


\section{Design og implementering}
\note{Fremstilling i bi-quad print}
\note{Argumenteret valg er OpAmp -> den bedste der var som SMD}
\note{Kort begrundelse for enkelt R i metoden og brug af op til 4 stk. C som proto type -> hvordan ville det se ud hvis kun brugte den nærmeste C i serien og hvor stor indflydelse vil det have. Måske Sensitivitesmetoden eller H(jw) -> var det er klogt valg og måske lidt overdrevet.}

\section{Rekonstruktions filter}
\note{Hvorfor anvendes et tilsvarende AA filter på udgangen - lidt teori her}

