\section{Opsummering}

\hlh{Noget i denne stil eller er det helt hen i vejret?..}
For helt at undgå aliasing effekten ville det kræve en ekstrem stor filter orden, for at overholde specifikationer.
Derfor er det valgt at der skal anvendes et 6. ordens anti-aliasing filter for herefter at analysere effekten af det.

Ved nærmere analyse fandtes et 6. ordens Chebychev 1 filter til at klare sig bedst inden
for kravende. Det viste sig at have en aliasing effekt på op til $5,4\si\percent$ ved
filterets knækfrekvens på $18\si\kilo\hertz$.

Ud fra det er der blevet designet et Sallen-Key aktivt lav-pas filter, som også er implementeret på et prototype print.

Der skal også anvendes et rekonstruktions filter til at udglatte signalet der bliver genereret af DAC'en, dette er valgt til at være det samme som anti-aliasing filteret da både indput og output tilhører samme frekvens spektrum.

Anti-alias filteret er dermed designet og implementeret efter planen, og overholder de stillede krav.

Udover det er der blevet produceret et indgangs led til at offsette indgangs-signalet til et niveau både filteret og mikrocontrolleren kan arbejde med. Der er ligeledes lavet et udgangsled der fjerne dette offset igen.
