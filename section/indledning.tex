\chapter{Indledning}
\vspace*{0.5 cm}
%\emph{intro}
I det følgende kapitel fastligges projektes formål og problemformulering der danner grundlag for rapporten.
Rapporten afspejler og dokumentere det udførte projektarbejde på 4. semester diplomingeniør i Elektronik og Datateknik på Syddansk Universitet, Teknisk fakultet / Mærsk instituttet.
Rapporten danner grundlag for eksiminationen for semester projektet hvor arbejdet er udført i perioden 1. februar til 25. maj 2018.

\husk{JJ}{Skal måske uddybes lidt og perioden flyttes til Forord.}

\section{Formål}
Semesterets hovedemne omhandler \emph{Indlejrede systemer og signalbehandling}, hvor faglighederne \emph{Analoge filtre og signaler}, \emph{Digital signalbehandling}, \emph{Embedded programmering} og \emph{Operativsystemer} er inkluderet.
Ud fra projektbeskrivelsen\footnote{ref. til  semesterprojekt beskrivelse her...} er formålet\textit{ "at skabe en applikation, der gør noget meningsfuldt, således at semesterets fire fagligheder er dækket ind"}.

\husk{JJ}{Find overgang mellem formål og problemformulering}

\section{Problemformulering}
Formålet med projektet er at undersøge hvordan lyd kan behandles og manipuleres ved at bruge digital signalbehandling i et embedded miljø.
Løsningen skal opbygges som en modulær løsning, således at nye "lydmoduler" kan efter udvikles.
Styringen af modulerne skal kunne forgå igennem EMP boardet - her tænkes det at kunne anvende HID styringen dette board stiller til rådighed og som sammen LCDen skal udgøre en platform til UI'en. 

\husk{JJ}{Husk at få indelt problemformulering i 3 grupper - Overodnet (motivation), Konkret og specifik (Udkast i listen).}


\begin{itemize}
	\item Er det muligt at lave et modulært audio effekt system der kan leve op til de fastsatte krav ? 

	\begin{itemize}
		
		\item Realtids lydbehandling 
		\item Digital filter design
		\item Analog filter design
		\item OS krav / design
		\item Test krav og eftervisning af forventede
	\end{itemize}
\end{itemize}
\husk{JJ}{Formulering af hoved- og underspørgsmål tilrettes.}

\section{Projektafgrænsning}
\begin{itemize}
	\item Produktet er ikke udviklet som et slutprodukt.
\end{itemize}

\section{Kravspecifikation} 
Kravspecikationerne for projektet er fastsat, dels ud fra kravet til selve projektbeskrivelsen og dels for at efterligne projekter udenfor det akademiske miljø.

\begin{itemize}[noitemsep]
	\item Prototypen udvikles på Tiva™ C Series TM4C123G LaunchPad Evaluation Kit \cite{spmt281a}.
	\item Tiva™ TM4C123G serie Microcontroller anvendes \cite{spmu296}.
	\item FreeRTOS anvendes som embedded kernel.
	\item Kildekoden er en del af produktet og skal overholde \textit{EMP C Code Standard}\cite{emp-c}.
	\item Der anvendes en samplingsfrekvens på $F_s = 44.1 \si{\kilo\hertz}$ stereo.
\end{itemize}


\section{Løsningsmodel}

\section{Proces- og arbejdsmetode}
