\chapter*{Forord}\label{chap:forord}
\addcontentsline{toc}{chapter}{Forord}
Dette projekt er udarbejdet af seks ingeniør-studerende på 4. semester som studerer Elektronik og Datateknik på Syddansk Universitet. 
Projektet er udført i perioden d. 1. februar til d. 25. maj 2018. 
Projektet er udført under vejledning af Ib Refer.

\subsection{Læsevejledning}
Rapporten bør læses fra start til slut som et samlet og sammenhængende værk. 
Som udgangspunkt antages det, at læseren har et fagligt kompetenceniveau som en studerende på 4. semester med linjefag indenfor elektronik og datateknik på syddansk universitet.

\subsection{Typografiske konventioner}
Her er en kort oversigt over de typografiske konventioner der anvendes i denne rapport.

\bigskip

\begin{tabular}{l p{0.6\linewidth}}
	\textit{Kursiv tekst}			& Angiver filnavne i den tilhørende kodebase samt fremhævelse af ord eller fagudtryk. \\
	\textbf{Fed tekst}				& Bruges til a fremhæve produkt eller system specifikke betegnelser.\\
	\texttt{Konstant brede tekst}	& Anvendes til kildekode eksempler. Ligeledes anvendes afgrænsende områder.\\
	\emph{Fremhævet tekst}		    & Bliver brugt når der gives en kort introduktion til hvert kapitel.\\
\end{tabular}

\subsection{Typografi}
Rapporten er fremstillet i \LaTeX - Memoir, sat i 11 pt. Computer Modern.\\
Antal sider : \pageref{LastPage}