\chapter*{Forord}\label{chap:forord}
\addcontentsline{toc}{chapter}{Forord}
Dette projekt er udarbejdet af seks ingeniør-studerende på 4. semester for Elektronik og Datateknik på Syddansk Universitet. 
Projektet er udført i perioden d. 1. februar til d. 25. maj 2018. 
Projektet er udført under vejledning af Ib Refer. Gruppen takker for støtten gennem forløbet. 

\husk{NOTE}{I nogle rapporter ønsker forfatterne at indsætte et forord. Dette kan være i form af en tak til
	samarbejdspartnere, en dedikation af projektet til bestemte personer, eller fordi de ønsker at
	redegøre for ændrede forhold i forbindelse med rapporten.
	Et forord er ikke en obligatorisk del af et projekt. Det er som navnet siger ”før ordet”, dvs. før selve
	rapporten og kan ofte udelades.}

\subsection{Læsevejledning}
Rapport er bør læses fra start til slut som et samlet og sammenhængende værk. 
Som udgangspunkt antages det, at læseren har et fagligt kompetenceniveau som en studerende på 4. semester med linjefag indenfor elektronik og datateknik.

\subsection{Typografiske konventioner}
Her er en kort oversigt over de typografiske konventioner der anvende i denne rapport\\
\begin{tabular}{l p{0.6\linewidth}}
	\textit{Kursiv tekst}			& Angiver filnavne i den tilhørende kodebase samt fremhævelse af ord eller fagudtryk. \\
	\textbf{Fed tekst}				& Bruges til a fremhæve produkt eller system specifikke betegnelser.\\
	\texttt{Konstant brede tekst}	& Anvendes til kildekode eksempler. Ligeledes anvendes afgrænsende områder.\\
	\emph{Fremhævet tekst}		    & Bliver brugt når der gives en kort introduktion til hvert kapitel.\\
\end{tabular}

\subsection{Typografi}
Rapporten er fremstillet i \LaTeX - Memoir, sat i 11 pt. Computer Modern.\\
Antal sider : \pageref{LastPage}