\section{FreeRTOS}
\note{Kort intro til hvad FreeRTOS er og hvilken funktionalitet det stiller tilrådighed}


\section{Preemptive schedulering}
FreeRTOS er bygget på en prioritetsbaseret preemptive scheduleringsalgoritme.\newline
Når et operativ system opererer med en preemptive scheduleringsalgoritme kan kørende processer preemptes - blive stoppet - og skiftet ud med en anden proces.\newline
Det kan f.eks. være at en proces der har ventet på en I/O device får tilgang til denne.
Scheduleren vil så skifte den nuværende kørende proces ud og skifte den hidtil ventende proces ind så den kan køres.
Dette gør scheduleren via et context switch.\newline
Når et context switch sker gemmes ''konteksten`` af den nuværende task i en process control block (PCB), og ydermere sker der et state restore, hvor informationen i PCBen af den task, der skal skiftes til hentes.
Det som bliver gemt i PCBen er værdierne i CPU registrene (Program counter, etc.) og anden vigtig operativ systemsinformation.\newline
Prioritetsbaseret skeduleringsalgoritmer tildeler alle tasks en prioritet som er baseret på taskens vigtighed.\newline
FreeRTOS er et real-time operativ system og et primært formål ved real-time operativ systemer er at give et respons på begivenheder indenfor en vis deadline.
FreeRTOS skeduleringsalgoritme sørger så for at den task med højest prioritet bliver givet processortid.

\section{Interrupt håndtering}

\note{Interrupt execution diagram og opsætning} 

\section{Sampling af lydsignal igennem ADC}


\section{Håndtering af brugergrænseflade og human input interface}

\section{Modulær opbygning af effekter}

\section{Genskabelse af signal via DAC igennem SPI}

