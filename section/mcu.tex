\section{FreeRTOS}
\note{Kort intro til hvad FreeRTOS er og hvilken funktionalitet det stiller tilrådighed}
Som indlejret styresystem benyttes FreeRTOS. 
FreeRTOS er et open source real-tids styresystem til indlejrede systemer, som er blevet en industriel standard. 
Styresystemet er valgt, da det er simpelt at gå til. 
Det er desuden primært skrevet i programmeringssproget C, som også benyttes i projektet. 
FreeRTOS benytter preemptive schedulering til at administrere CPU-tiden mellem tasks. 

\section{Preemptive schedulering}
FreeRTOS er bygget på en prioritetsbaseret preemptive scheduleringsalgoritme.\newline
Når et operativ system opererer med en preemptive scheduleringsalgoritme kan kørende processer preemptes - blive stoppet - og skiftet ud med en anden proces.\newline
Det kan f.eks. være at en proces der har ventet på en I/O device får tilgang til denne.
Scheduleren vil så skifte den nuværende kørende proces ud og skifte den hidtil ventende proces ind så den kan køres.
Dette gør scheduleren via et context switch.\newline
Når et context switch sker gemmes ''konteksten`` af den nuværende task i en process control block (PCB), og ydermere sker der et state restore, hvor informationen i PCBen af den task, der skal skiftes til hentes.
Det som bliver gemt i PCBen er værdierne i CPU registrene (Program counter, etc.) og anden vigtig operativ systemsinformation.\newline
Prioritetsbaseret skeduleringsalgoritmer tildeler alle tasks en prioritet som er baseret på taskens vigtighed.\newline
FreeRTOS er et real-time operativ system og et primært formål ved real-time operativ systemer er at give et respons på begivenheder indenfor en vis deadline.
FreeRTOS skeduleringsalgoritme sørger så for at den task med højest prioritet bliver givet processortid.

\section{Interrupt håndtering}
\note{Interrupt execution diagram og opsætning} 
Når det indgående lydsignal skal samples gennem ADC'en, skal samplingen foregå periodisk på nøjagtigt det samme tidspunkt i hver periode.
For at sikre, at mikrocontrolleren sætter alle andre opgaver på pause, og begynder at sample på det korrekte tidspunkt, implementeres samplingen i en interrupt service routine, også kaldet ISR.
Rutinen indstilles med den højest mulige interrupt prioritet. 
Timingen af ISR'en er styret af Timer 3. 
Timeren er implementeret som en 16-bit timer i periodic timer mode, edge-count mode og inverted PWM mode.
Ved start hentes timerens start value ind i et tælleregister. 
Sample-frekvensen og CPU frekvensen styrer værdien. 
\begin{equation}
	\text{Timer start value register} = \frac{\text{CPU'ens frekvens}}{\text{Sample-frekvens}} = \frac{80\text{MHz}}{44,1\text{kHz}} \simeq 1814
\end{equation}
I periodic timer mode vil timeren dekrementere fra tælleregisteret, som automatisk henter timer start værdien igen, og begynder forfra når værdien når nul. 
Når værdien når nul, kaldes den ISR som sikrer at lydsignalet bliver samplet. \newline

Timeren benyttes til at generere et PWM-signal, hvis duty cycle er styret af indgangssignalet. 
% match value loades ind
% Hvor kommer PWM ud?
% Skal det efter ADC afsnit?

\section{Sampling af lydsignal igennem ADC}

\section{Håndtering af brugergrænseflade og human input interface}

\section{Modulær opbygning af effekter}

\section{Genskabelse af signal via DAC igennem SPI}

