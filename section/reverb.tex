\chapter{Reverb effektmodul}\label{chap:reverb}
Reverb, rumklang, er resultatet af lydbølger som bliver reflekteret tilbage af samtlige overflader med forskellige vinkler.\newline
Rumklang er resultatet af lydbølger som bliver reflekteret af samtlige overflader i et rum, derved bygges der en masse reflektioner op hvis amplitude falder mod $0$ som de bliver absorberet af overfladerne i rummet.\newline
Der er to primære metoder for implementeringen af en digital reverb effekt.
Der er convolution (foldnings) reverben og den algoritmiske reverb.
\subsection{Convolution reverb}
Reverberation, rumklang, er en tidsinvariant effekt, hvilket betyder at det ikke har nogen betydning, hvornår en tone bliver spillet, det vil ultimativt resultere i præcis den samme reverberation. \newline
Med tidsinvariante systemer kan de karakteriseres ved deres impulsrespons.
Convolution reverbs virker ved at lave en foldning af rummets impulsrespons og det lydsignal der sættes på indgange n til reverben.\newline
Dette skaber en realistisk rumklangseffekt, fordi impulsen i dette tilfælde vil være en lyd som holder samme energiniveau ved alle frekvenser.
Efter impulsen bliver spillet vil den blive reflekteret rundt i rummet.
Nogle af reflektionerne møder mikrofonen med det samme mens andre bliver ført rundt i rummet og amplituden af signalerne går mod $0$ pga. overfladerne af objekterne i rummet absorbere energien fra dem.\newline
Multiplikering af hver punkt af impulsresponset med amplituden af samplet giver så rummets respons til den sample.
Dette gøres for hver sample af inputtet og giver overlappende responser som så adderes og resulterer i rumklang.
En ulempe ved convolution reverbs er, at der skal mange beregninger til for at få resultatet.
Hvert sample skal individuelt multipliceres med hvert sample af impulsresponset og adderes.
Hvis der haves $N$ samples og impulsresponset er $M$ samples lang skal der udføres $N+M$ multiplikationer og additioner.
F.eks. hvis der haves et impulsrespons på 1 sekund og der samples med $44.1\si{kHz}$, skal der udføres næsten 2 milliarder multiplikationer og additioner i sekundet.
Antallet af multiplikationer og additioner kan dog reduceres drastisk ved at arbejde i frekvensdomænet i stedet for, da foldning i frekvensdomænet er multiplikation.

Fordelen ved convolution reverbs er at ethvert rum i verden kan imiteres, hvis impulsresponset for det valgte rum haves.\newline
Derudover kan man opfinde rum ved at syntetisere et impulsrespons.

\subsection{Algoritmisk reverb}
En algoritmisk reverb virker ved at bruge flere forskellige delays og feedback loops til at opbygge en serie af ekkoer, som dør ud over tid.
Det er sammensætningen af de basale byggeblokke som giver karakteristikken på rummet der emuleres.\newline
Et eksempel på en simpel algoritmist reverb effekt er all-pass filteret.
%insert billede af all-pass filter
Her bliver samplet feeded forward til outputtet så lyden bliver spillet med det samme.
Derudover smides samplet ind i en delay buffer.
Rumklangen skabes så af samples fra delay bufferen, som fungerer som opbygningen af ekkoer og feedback loopet som agerer som absorptionen for at skabe aftagende ekkoer.





\section{Implementering af reverb}
\subsection{Kode}
\lstset{frame=tb,
	language=c,
	aboveskip=3mm,
	belowskip=3mm,
	showstringspaces=false,
	columns=flexible,
	basicstyle={\small\ttfamily},
	numbers=none,
	numberstyle=\tiny\color{gray},
	keywordstyle=\color{blue},
	commentstyle=\color{dkgreen},
	stringstyle=\color{mauve},
	breaklines=true,
	breakatwhitespace=true,
	tabsize=3,
	texcl=true
}
\begin{lstlisting}
void mod_reverb_effekt( fp_sample_t *in, fp_sample_t *out)
{
	if(is_sw1_pressed)
	{
		const float in_gain = -0.25;
		const float fb_gain = -0.05;
		const uint16_t delay = 2000;

	fp_sample_t fp_sample;
	sample_buffer_get(&fp_sample);

	fp_sample.left_fp32 = ((in->left_fp32 * in_gain) + ((in->left_fp32 + fp_sample.left_fp32) * fb_gain));
	fp_sample.right_fp32 = ((in->right_fp32 * in_gain) + ((in->left_fp32 + fp_sample.right_fp32) * fb_gain));

	sample_buffer_put_z(&fp_sample, delay);

	out->left_fp32 = in->left_fp32;
	out->right_fp32 = in->right_fp32;
	}
	else
	{
	out->left_fp32 = in->left_fp32;
	out->right_fp32 = in->right_fp32;
	}
}
\end{lstlisting}

%Koden er implementeringen af et all-pass filter.
%Først hentes en sample ind fra sample bufferen.
%Input samplet bliver ganget med feedback gainet og input gainet og lægges DELAY pladser frem i sample bufferen.\newline
%Der opbygges en masse ekkoer, da funktionen bliver kaldt hver gang der kommer et sample ind.\newline
%Disse ekkoer der ligger på i sample bufferen bliver lagt sammen med 

