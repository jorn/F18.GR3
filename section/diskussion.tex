\chapter{Diskussion og vurdering}\label{kap:diskussion}
\vspace*{.5cm}
 
\note{ved ikke hvordan dette afsnit skal opbygges}

En af de begrænsende faktorer ved implementering af effektmodulerne er den korte tid, der er til at foretage beregninger mellem samples bliver taget til de skal ud igen.
Den begrensede tid kan skabe et problem for store beregninger der kan tage lang tid, hvis en sådan beregning på signalet ikke bliver færdiggjort før samplet skal ud igen bliver signalbehandlingen ikke korrekt udført, eller det kan resultere i et delay af signalet, hvilket ikke ønskes. \husk{Emil}{jeg her helt bestemt ikke tilfreds med denne sætning...}
Af hensyn til dette er det vigtigt at lave beregninger så hurgtigt og effiktivt som muligt.
Ved både Echo- og Reverbeffektmodulerne er dette tilfældet, da der ved hvert sample kun skal laves et ganske lille antal simple beregninger.
I Echoeffektmodulet er bliver et sample blot ganget med en gain/decay \note{hvad kalder vi det?} og derefter adderet til en værdig i en buffer.
Reverbeffekten opnås ved en lignende metode, det er næmlig valgt at benytte en algoritmisk reverb frem for convolution reverb, altså ved foldning.
Dette er valgt, selvom der i teorien ville kunne opnås en bedre, mere realistisk, reverbeffekt ved convolution reverb.
Valget er truffet, da det ville betyde at der skulle foretages flere beregninger imellem input og output samples.
Hvilket ville betyde, at der ville være en risiko for ikke at nå beregningerne, uden en stor mængde planlægning af beregningerne.
Denne metode har også behov for en større mængde hukommelse end den algoritmiske reverb effekt.
Dette er også en af grundende til, at dette valg blev truffet. 
Da det ende med en pladsmangel på microcontrolleren, når der skal ligge flere forskellige moduler på.
\note{er det det her vi leder efter i diskussionen?}
