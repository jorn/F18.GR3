\chapter{Diskussion og vurdering}\label{kap:diskussion}
\vspace*{.5cm}

Mange af de valg der er blevet taget ved design af de analoge filtre er en blanding af kompromiser. Der findes desværre ikke en endelig korrekt løsningsmodel og således har valgene været ret subjektive.
Ideelt ville et Bessel filter have givet den bedste gruppeløbetid, men det er samtidigt den design metode der har mest ringe dæmpning.

Valget af modulet bi-quad filter PCB, var praktisk som prototype, men medførte også en del støj pga det store areal og de mange forbindelsesledninger.

Problemerne under dimensionering af filtrene medførte, at de teoretisk frembragte værdier ikke kunne anvendes direkte. Det var ikke nogen hindring at eftervise og sammenholde den teoretiske del med den praktiske ved at anvende tabelværdier i stedet.
\\
\\
%MCU
For at kunne håndtere en korrekt sampling af signalet med præcis timing der bl.a. fjerner jitter fejl i lydgengivelse, blev det valgt at sampling ISR'en lå uden for FreeRTOS' task-håndtering. 
Dette medførte at det kun var den resterende tid (imellem hver sample) hvor FreeRTOS havde mulighed for at udføre de underliggende tasks.
Implementeringen af de mellemliggende ISR'er til håndtering af digiswitch og keypad var underlagt en lavere prioritet for at kunne sende data igennem køer styret af FreeRTOS. Det kunne således give andledning til starvation. 
Det var derfor vigtigt at få fastlagt en passende taskdelay og taskprioritering.    
\\
Det er ikke alle dele af den udviklede kode til projektet der er blevet helt færdiggjort inden afleveringsfristen.
Den overordnede funktionalitet er på plads, selv om visse funktionsvalg, som fx aktivering og deaktivering af moduler, kan kræve en ændring i kildekoden.
På samme måde er en 100\% løs kobling imellem moduler og kerne software heller ikke blevet opnået, men ville være mulig med mere udvikling - grundideerne er på plads. 
\\
\\
Den algoritmiske implementering af de digitale effekter medførte, at en anden tilgang til buffer princippet var nødvendig.
Kaskadekobling af denne type effekter var ikke mulig med én enkelt buffer, og begrænset hukommelse i projektets mikrocontroller betød, at flere buffere med tilstrækkelig længde ikke var en mulighed.

En betydeligt mere realistisk rumklangs/ekko effekt kunne have været opnået med et ''convolution reverb``, dette ville dog have krævet en bedre mikrocontroller med højere clockfrekvens og mere plads til at gemme data og der blev derfor i stedet valgt at anvende algoritmiske løsninger.

Den valgte FIR filter design metode viste sig at fungere med de ressourcer der er til rådighed. 
En af ulemperne ved \textit{frequency samplings} metoden, er ripple dannelse imellem de samplede frekvenser.



