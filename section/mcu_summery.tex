\section{Opsummering}

\subsection{FreeRTOS}


\subsection{Preemptive schedulering}


\subsection{Løsningsimplementering og task model}


\subsection{Intterrupt eksekvering og task skedulering}
Det indgående lydsignal samples periodisk på samme tidspunkt i hver periode. 
Det gøres ved hjælp af en ISR med højeste prioritet, som også er prioriteret over tasks. 
Timingen af ISR er kontrolleret af Timer 3.
Timeren er implementeret således, at der opnås en samplefrekvens på 44,1kHz. 
Alle andre processor opgaver skal derfor foregå i tiden mellem hver gang ISR sample handler rutinen kører. 

\subsection{Sampling af lydsignal igennem ADC}
Til sampling af signalet benyttes to 12-bit ADC moduler. 
Samplingen startes ved et processor event, som foregår i ISR sample handler rutinen. 
Det samplede data indeholder et offset fra input stage, som efter samplingen fjernes. 

\subsection{Modulær opbygning af effekter}


\section{Generering af PWM-signal til DAC}


\subsection{Genskabelse af signal via DAC igennem SPI}


\subsection{LCD Driver}


\subsection{Håndtering af brugergrænseflade og human input interface}
%Et modulært menu system er nødvendigt for at effektmoduler kan tilføres helt uafhængigt af systemet, det lykkedes dog ikke at producere et sådant system men det skyldtes dog blot tidsmangel og kunne nemt implementeres i det nuværende menu system i form af en software patch, derfor blev dette ned prioriteret så deadline kunne overholdes. 

Et menu system er udviklet til enheden, bestående af en datastruktur i form af en linked list og en state machine der skriver menuen til displayet.
Et modulært menu system er nødvendigt for at effektmoduler kan tilføres helt uafhængigt af systemet, dette er dog blevet nedpriorieteret grundet tidspres, og er derfor ikke gjort.
Input fra brugeren sker gennem to switches på TIVA-boardet, samt en drehimpulsgeber på EMP-boardet.
Der bliver pullet på switchesne og events bliver genereret af en statemachine og sendt ud til andre tasks, hvor rotation på drehimpulsgeberen trigger et interrupt der afgøre retningen af rotation.