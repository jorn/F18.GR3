\chapter{Konklusion} \label{kap:konklusion}

De analoge filtre i løsningen blev implementeret som 6. ordens Chebyshev Type I filtre. Filtre skulle bruges som båndbreddebegrænsning i et anti-aliasing filter på indgang til mikrocontrolleren og et symmetrisk filterdesign blev anvendt som rekonstruktionsfilter på udgangen.
Kravene til disse filtre viste sig at være ret subjektive, så derfor blev konsekvenserne af de endelige valg analyseret og kortlagt i stedet. 
\\
For at kunne implementere de digitale effekter, skulle et \textit{signal-flow} diagram anvendes og en passende buffer arkitektur var nødvendig. 
Begrænsede systemressourcer var betydeligt mere afgørende for implementeringsvalget end realtidskravet. 
\\	
Den valgte mikrocontroller viste sig tilstrækkelig for at kunne implementere et realtidssystem og FreeRTOS stillede en lang række funktionaliteter til rådighed, der viste sig passende for løsningen. 
\\
Således var det overordnet muligt at fremstille et modulært effekt system, der levede op til de stillede krav.
